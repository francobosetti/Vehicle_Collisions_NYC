\documentclass[10pt]{article}\usepackage[]{graphicx}\usepackage[]{xcolor}
% maxwidth is the original width if it is less than linewidth
% otherwise use linewidth (to make sure the graphics do not exceed the margin)
\makeatletter
\def\maxwidth{ %
  \ifdim\Gin@nat@width>\linewidth
    \linewidth
  \else
    \Gin@nat@width
  \fi
}
\makeatother

\definecolor{fgcolor}{rgb}{0.345, 0.345, 0.345}
\newcommand{\hlnum}[1]{\textcolor[rgb]{0.686,0.059,0.569}{#1}}%
\newcommand{\hlsng}[1]{\textcolor[rgb]{0.192,0.494,0.8}{#1}}%
\newcommand{\hlcom}[1]{\textcolor[rgb]{0.678,0.584,0.686}{\textit{#1}}}%
\newcommand{\hlopt}[1]{\textcolor[rgb]{0,0,0}{#1}}%
\newcommand{\hldef}[1]{\textcolor[rgb]{0.345,0.345,0.345}{#1}}%
\newcommand{\hlkwa}[1]{\textcolor[rgb]{0.161,0.373,0.58}{\textbf{#1}}}%
\newcommand{\hlkwb}[1]{\textcolor[rgb]{0.69,0.353,0.396}{#1}}%
\newcommand{\hlkwc}[1]{\textcolor[rgb]{0.333,0.667,0.333}{#1}}%
\newcommand{\hlkwd}[1]{\textcolor[rgb]{0.737,0.353,0.396}{\textbf{#1}}}%
\let\hlipl\hlkwb

\usepackage{framed}
\makeatletter
\newenvironment{kframe}{%
 \def\at@end@of@kframe{}%
 \ifinner\ifhmode%
  \def\at@end@of@kframe{\end{minipage}}%
  \begin{minipage}{\columnwidth}%
 \fi\fi%
 \def\FrameCommand##1{\hskip\@totalleftmargin \hskip-\fboxsep
 \colorbox{shadecolor}{##1}\hskip-\fboxsep
     % There is no \\@totalrightmargin, so:
     \hskip-\linewidth \hskip-\@totalleftmargin \hskip\columnwidth}%
 \MakeFramed {\advance\hsize-\width
   \@totalleftmargin\z@ \linewidth\hsize
   \@setminipage}}%
 {\par\unskip\endMakeFramed%
 \at@end@of@kframe}
\makeatother

\definecolor{shadecolor}{rgb}{.97, .97, .97}
\definecolor{messagecolor}{rgb}{0, 0, 0}
\definecolor{warningcolor}{rgb}{1, 0, 1}
\definecolor{errorcolor}{rgb}{1, 0, 0}
\newenvironment{knitrout}{}{} % an empty environment to be redefined in TeX

\usepackage{alltt}
\usepackage[left=1in,right=1in,top=1in,bottom=1in]{geometry}
\usepackage{amsmath}
\usepackage{hyperref}
%\SweaveOpts[concordance=TRUE]
\IfFileExists{upquote.sty}{\usepackage{upquote}}{}
\begin{document}

\begin{titlepage}
    \centering
    
    { \Huge \textbf{Final Project Report Draft 2} \par} 
    \vspace{5mm}
    { \huge \textbf{Motor vehicle colisions in NYC} \par} 
    \vspace{3mm}
    { \Large Group 1 (STAT 605 - S1)\par}  
    
    \vspace{5mm}
    \begin{center}
    \line(1,0){425}
    \end{center}
    
    \vspace{5mm}

    \vspace{15mm}
    { \large Authors: \par }
      \vspace{3mm}
    { \Large \textbf{Astha Singh} - \texttt{as677@rice.edu} \par}
    { \Large \textbf{Franco Bosetti} - \texttt{fb57@rice.edu}\par}
    { \Large \textbf{Louis Clarke} - \texttt{lc160@rice.edu}\par}
    { \Large \textbf{Lindsey Russ} - \texttt{ltr1@rice.edu}\par}
    
    
    \vspace{5mm}
    
    \vspace{5mm}

\end{titlepage}

\newpage


\tableofcontents

\newpage

\section{Introduction}

Motor vehicle collisions are a major concern in major populated urban areas like New York City (NYC), where millions of people use diverse transportation systems every day. Understanding these incidents through data analysis can help us identify patterns and make safety improvements. At the same time, social media platforms such as Twitter have become spaces where people share their real-time information about traffic incidents and congestion. This project will analyze NYC motor vehicle collision data alongside Twitter traffic discussions, while also incorporating roadway speed limit data and detailed vehicle-level records to examine how roadway conditions, vehicle characteristics, and public reporting intersect with officially documented traffic incidents.

\section{Dataset Description}
Primary Dataset: NYC Motor Vehicle Collisions 

(\url{https://data.cityofnewyork.us/Public-Safety/Motor-Vehicle-Collisions-Crashes/h9gi-nx95/about_data})

Our primary dataset contains police-reported motor vehicle collision records from New York City, maintained by the NYPD through their TrafficStat initiative. The dataset includes 2.21 million collision records with 29 variables.


Source: NYC OpenData (2025)


Key variables include:
\begin{itemize}
  \item Location \& Time: Borough, coordinates (latitude \& longitude), street names, zip codes, crash date/time
  \item Casualties: Injuries and fatalities by category (pedestrians, cyclists, motorists)
  \item Contributing Factors: Up to five causal factors per collision
  \item Vehicle Types: Vehicle type classifications (ATV, bicycle, car/SUV, e-bike, e-scooter, truck/bus, motorcycle, other) for each vehicle involved
\end{itemize}


Preprocessing: We prepared the dataset for analysis by converting crash dates to a standard date format, organizing borough names as categorical variables, and creating additional time-based variables to explore patterns across years and months. 

Secondary Dataset: Twitter Traffic Classifications (\url{https://data.mendeley.com/datasets/c3xvj5snvv/1})

Our secondary dataset contains classified tweets collected from Twitter's API, with each tweet categorized by its traffic-related content.


Source: Research dataset by Sina Dabiri (2018)


Variables include:
\begin{itemize}
  \item Tweet ID
  \item Classification label: 
  \item Non-Traffic (Class 0): General tweets unrelated to traffic
  \item Traffic Incident (Class 1): Reports of crashes, breakdowns, road closures
  \item Traffic Conditions (Class 2): Congestion reports, traffic advisories, flow conditions
  \item Tweet text content
\end{itemize}


Preprocessing: We combined the training and test datasets into one dataset since we don’t plan to run a classification model. We renamed variables for clarity and ensured consistency in column naming. Our next step is cleaning the dataset and the text data to prepare for sentiment analysis and linking patterns with NYC collision data.

Tertiary Dataset: NYC Speed Limits (\url{https://data.cityofnewyork.us/Transportation/VZV_Speed-Limits/5mad-ntua/about_data}) 

Our tertiary dataset provided posted speed limits across NYC streets, maintained under the Vision Zero initiative. This dataset will allow us to explore whether high-speed areas correlate with higher accident frequency and severity.


Source: NYC OpenData (2025)


Variables include:
\begin{itemize}
  \item Segment Geometry: street segment represented as geographic coordinates (longitude & latitude)
  \item Posted Speed Limit: legal speed limit value for each street segment
  \item Street Name
  \item Borough
\end{itemize}

Preprocessing: We extracted latitude and longitude from the geometric strings and rounded them to 4 decimals to match with collision coordinates. This allowed us to link crash locations to nearby street segments. We also filtered roads with speed limits above 40 mph to compare crashes on high-speed vs. lower-speed roads. 

Quaternary Dataset: Motor Vehicle Collisions - Vehicles (\url{https://data.cityofnewyork.us/Public-Safety/Motor-Vehicle-Collisions-Vehicles/bm4k-52h4/about_data}) 

Our quaternary dataset gives vehicle-level details for each crash in NYC. Unlike the primary collisions dataset, where each row represents a crash, here each row represents a single vehicle involved. This will give us insights into how vehicle type, driver demographics, and pre-crash behavior contribute to accidents.

Source: NYC OpenData (2025)


Variables include:
\begin{itemize}
  \item Collision information 
  \item Vehicle details 
  \item Driver information
  \item Crash circumstances 
  \item Contributing factors 
\end{itemize}

Preprocessing: We plan to link this dataset with the primary collisions dataset using the collision_id key. This will also expand our crash-level analysis with vehicle-level and driver-level details, such as identifying trends in which vehicle types are most frequently involved in accidents and which driver factors contribute most to severe outcomes.  

\newpage


\section{Visualizations}

Prepare the data for processing:

\begin{knitrout}
\definecolor{shadecolor}{rgb}{0.969, 0.969, 0.969}\color{fgcolor}\begin{kframe}
\begin{alltt}
\hlcom{# NYC Motor Collisions Dataset}
\hldef{nyc} \hlkwb{<-} \hlkwd{read.csv}\hldef{(}\hlkwc{file} \hldef{=} \hlsng{"data/Motor_Vehicle_Collisions_-_Crashes.csv"}\hldef{)}
\end{alltt}


{\ttfamily\noindent\color{warningcolor}{\#\# Warning in file(file, "{}rt"{}): cannot open file 'data/Motor\_Vehicle\_Collisions\_-\_Crashes.csv': No such file or directory}}

{\ttfamily\noindent\bfseries\color{errorcolor}{\#\# Error in file(file, "{}rt"{}): cannot open the connection}}\begin{alltt}
\hldef{nyc}\hlopt{$}\hldef{CRASH.DATE} \hlkwb{<-} \hlkwd{as.Date}\hldef{(nyc}\hlopt{$}\hldef{CRASH.DATE,} \hlkwc{format} \hldef{=} \hlsng{"%m/%d/%Y"}\hldef{)}
\end{alltt}


{\ttfamily\noindent\bfseries\color{errorcolor}{\#\# Error: object 'nyc' not found}}\begin{alltt}
\hldef{nyc}\hlopt{$}\hldef{BOROUGH} \hlkwb{<-} \hlkwd{factor}\hldef{(nyc}\hlopt{$}\hldef{BOROUGH)}
\end{alltt}


{\ttfamily\noindent\bfseries\color{errorcolor}{\#\# Error: object 'nyc' not found}}\begin{alltt}
\hldef{nyc}\hlopt{$}\hldef{YEAR_MONTH} \hlkwb{<-} \hlkwd{format}\hldef{(nyc}\hlopt{$}\hldef{CRASH.DATE,} \hlsng{"%Y-%m"}\hldef{)}
\end{alltt}


{\ttfamily\noindent\bfseries\color{errorcolor}{\#\# Error: object 'nyc' not found}}\begin{alltt}
\hldef{nyc}\hlopt{$}\hldef{YEAR} \hlkwb{<-} \hlkwd{format}\hldef{(nyc}\hlopt{$}\hldef{CRASH.DATE,} \hlsng{"%Y"}\hldef{)}
\end{alltt}


{\ttfamily\noindent\bfseries\color{errorcolor}{\#\# Error: object 'nyc' not found}}\begin{alltt}
\hldef{nyc}\hlopt{$}\hldef{HOUR} \hlkwb{<-} \hlkwd{as.numeric}\hldef{(}\hlkwd{sub}\hldef{(}\hlsng{"\textbackslash{}\textbackslash{}:.*"}\hldef{,} \hlsng{""}\hldef{, nyc}\hlopt{$}\hldef{CRASH.TIME))}
\end{alltt}


{\ttfamily\noindent\bfseries\color{errorcolor}{\#\# Error: object 'nyc' not found}}\begin{alltt}
\hlcom{# Twitter Dataset}
\hldef{twitter_train} \hlkwb{<-} \hlkwd{read.csv}\hldef{(}\hlsng{"data/c3xvj5snvv-1/1_TrainingSet_3Class.csv"}\hldef{,} \hlkwc{header}\hldef{=}\hlnum{FALSE}\hldef{)}
\end{alltt}


{\ttfamily\noindent\color{warningcolor}{\#\# Warning in file(file, "{}rt"{}): cannot open file 'data/c3xvj5snvv-1/1\_TrainingSet\_3Class.csv': No such file or directory}}

{\ttfamily\noindent\bfseries\color{errorcolor}{\#\# Error in file(file, "{}rt"{}): cannot open the connection}}\begin{alltt}
\hldef{twitter_test} \hlkwb{<-} \hlkwd{read.csv}\hldef{(}\hlsng{"data/c3xvj5snvv-1/1_TestSet_3Class.csv"}\hldef{,} \hlkwc{header}\hldef{=}\hlnum{FALSE}\hldef{)}
\end{alltt}


{\ttfamily\noindent\color{warningcolor}{\#\# Warning in file(file, "{}rt"{}): cannot open file 'data/c3xvj5snvv-1/1\_TestSet\_3Class.csv': No such file or directory}}

{\ttfamily\noindent\bfseries\color{errorcolor}{\#\# Error in file(file, "{}rt"{}): cannot open the connection}}\begin{alltt}
\hlkwd{names}\hldef{(twitter_train)} \hlkwb{<-} \hlkwd{c}\hldef{(}\hlsng{"class"}\hldef{,} \hlsng{"tweet_id"}\hldef{,} \hlsng{"tweet_text"}\hldef{)}
\end{alltt}


{\ttfamily\noindent\bfseries\color{errorcolor}{\#\# Error: object 'twitter\_train' not found}}\begin{alltt}
\hlkwd{names}\hldef{(twitter_test)} \hlkwb{<-} \hlkwd{c}\hldef{(}\hlsng{"class"}\hldef{,} \hlsng{"tweet_id"}\hldef{,} \hlsng{"tweet_text"}\hldef{)}
\end{alltt}


{\ttfamily\noindent\bfseries\color{errorcolor}{\#\# Error: object 'twitter\_test' not found}}\begin{alltt}
\hldef{twitter} \hlkwb{<-} \hlkwd{rbind}\hldef{(twitter_train, twitter_test)}
\end{alltt}


{\ttfamily\noindent\bfseries\color{errorcolor}{\#\# Error: object 'twitter\_train' not found}}\begin{alltt}
\hlcom{# Speed Limit Dataset}

\hldef{sp_limits} \hlkwb{<-} \hlkwd{read.csv}\hldef{(}\hlkwc{file} \hldef{=} \hlsng{"Project/data/speed_limits.csv"}\hldef{)}
\end{alltt}


{\ttfamily\noindent\color{warningcolor}{\#\# Warning in file(file, "{}rt"{}): cannot open file 'Project/data/speed\_limits.csv': No such file or directory}}

{\ttfamily\noindent\bfseries\color{errorcolor}{\#\# Error in file(file, "{}rt"{}): cannot open the connection}}\begin{alltt}
\hlcom{# Format the sp_limits latitude and longitude}
\hldef{splitted} \hlkwb{<-} \hlkwd{strsplit}\hldef{(sp_limits}\hlopt{$}\hldef{the_geom,} \hlsng{" "}\hldef{)}
\end{alltt}


{\ttfamily\noindent\bfseries\color{errorcolor}{\#\# Error: object 'sp\_limits' not found}}\begin{alltt}
\hldef{lats} \hlkwb{<-} \hlkwd{as.numeric}\hldef{(}\hlkwd{lapply}\hldef{(splitted,} \hlkwa{function}\hldef{(}\hlkwc{x}\hldef{)} \hlkwd{substr}\hldef{(x[}\hlnum{3}\hldef{],} \hlnum{1}\hldef{,} \hlkwd{nchar}\hldef{(x[}\hlnum{3}\hldef{])}\hlopt{-}\hlnum{1}\hldef{)))}
\end{alltt}


{\ttfamily\noindent\bfseries\color{errorcolor}{\#\# Error: object 'splitted' not found}}\begin{alltt}
\hldef{longs} \hlkwb{<-} \hlkwd{as.numeric}\hldef{(}\hlkwd{lapply}\hldef{(splitted,} \hlkwa{function}\hldef{(}\hlkwc{x}\hldef{)} \hlkwd{substr}\hldef{(x[}\hlnum{2}\hldef{],} \hlnum{3}\hldef{,} \hlkwd{nchar}\hldef{(x[}\hlnum{3}\hldef{]))))}
\end{alltt}


{\ttfamily\noindent\bfseries\color{errorcolor}{\#\# Error: object 'splitted' not found}}\begin{alltt}
\hldef{sp_limits}\hlopt{$}\hldef{latitude} \hlkwb{<-} \hldef{lats}
\end{alltt}


{\ttfamily\noindent\bfseries\color{errorcolor}{\#\# Error: object 'lats' not found}}\begin{alltt}
\hldef{sp_limits}\hlopt{$}\hldef{longitude} \hlkwb{<-} \hldef{longs}
\end{alltt}


{\ttfamily\noindent\bfseries\color{errorcolor}{\#\# Error: object 'longs' not found}}\begin{alltt}
\hldef{sp_limits}\hlopt{$}\hldef{location} \hlkwb{<-} \hlkwd{paste}\hldef{(}\hlsng{"("}\hldef{,} \hlkwd{round}\hldef{(lats,} \hlkwc{digits} \hldef{=} \hlnum{4}\hldef{),} \hlsng{", "}\hldef{,} \hlkwd{round}\hldef{(longs,} \hlkwc{digits} \hldef{=} \hlnum{4}\hldef{),} \hlsng{")"}\hldef{,} \hlkwc{sep} \hldef{=} \hlsng{""}\hldef{)}
\end{alltt}


{\ttfamily\noindent\bfseries\color{errorcolor}{\#\# Error: object 'lats' not found}}\begin{alltt}
\hlcom{# Add a rounded location value to the nyc dataset}
\hldef{has_long_lat} \hlkwb{<-} \hldef{nyc[}\hlopt{!}\hlkwd{is.na}\hldef{(nyc}\hlopt{$}\hldef{LONGITUDE)} \hlopt{& !}\hlkwd{is.na}\hldef{(nyc}\hlopt{$}\hldef{LATITUDE),]}
\end{alltt}


{\ttfamily\noindent\bfseries\color{errorcolor}{\#\# Error: object 'nyc' not found}}\begin{alltt}
\hldef{nyc_longs} \hlkwb{<-} \hldef{has_long_lat}\hlopt{$}\hldef{LONGITUDE}
\end{alltt}


{\ttfamily\noindent\bfseries\color{errorcolor}{\#\# Error: object 'has\_long\_lat' not found}}\begin{alltt}
\hldef{nyc_lats} \hlkwb{<-} \hldef{has_long_lat}\hlopt{$}\hldef{LATITUDE}
\end{alltt}


{\ttfamily\noindent\bfseries\color{errorcolor}{\#\# Error: object 'has\_long\_lat' not found}}\begin{alltt}
\hldef{has_long_lat}\hlopt{$}\hldef{round_location} \hlkwb{<-} \hlkwd{paste}\hldef{(}\hlsng{"("}\hldef{,} \hlkwd{round}\hldef{(nyc_lats,} \hlkwc{digits} \hldef{=} \hlnum{4}\hldef{),} \hlsng{", "}\hldef{,} \hlkwd{round}\hldef{(nyc_longs,} \hlkwc{digits} \hldef{=} \hlnum{4}\hldef{),} \hlsng{")"}\hldef{,} \hlkwc{sep} \hldef{=} \hlsng{""}\hldef{)}
\end{alltt}


{\ttfamily\noindent\bfseries\color{errorcolor}{\#\# Error: object 'nyc\_lats' not found}}\end{kframe}
\end{knitrout}

\subsection{Vehicle Collisions over Time}
By grouping the collisions into each month in which they occured, we can plot a line graph showing the number of collisions over the past 13 years.
\begin{knitrout}
\definecolor{shadecolor}{rgb}{0.969, 0.969, 0.969}\color{fgcolor}\begin{kframe}
\begin{alltt}
\hldef{monthly_counts} \hlkwb{<-} \hlkwd{table}\hldef{(nyc}\hlopt{$}\hldef{YEAR_MONTH)}
\end{alltt}


{\ttfamily\noindent\bfseries\color{errorcolor}{\#\# Error: object 'nyc' not found}}\begin{alltt}
\hldef{months} \hlkwb{<-} \hlkwd{names}\hldef{(monthly_counts)}
\end{alltt}


{\ttfamily\noindent\bfseries\color{errorcolor}{\#\# Error: object 'monthly\_counts' not found}}\begin{alltt}
\hldef{colspermonth} \hlkwb{<-} \hlkwd{as.numeric}\hldef{(monthly_counts)}
\end{alltt}


{\ttfamily\noindent\bfseries\color{errorcolor}{\#\# Error: object 'monthly\_counts' not found}}\begin{alltt}
\hldef{plot_months} \hlkwb{<-} \hlkwd{as.Date}\hldef{(}\hlkwd{paste0}\hldef{(months,} \hlsng{"-01"}\hldef{))}
\end{alltt}


{\ttfamily\noindent\bfseries\color{errorcolor}{\#\# Error in paste0(months, "{}-01"{}): cannot coerce type 'closure' to vector of type 'character'}}\begin{alltt}
\hlkwd{plot}\hldef{(plot_months, colspermonth,}
     \hlkwc{main}\hldef{=}\hlsng{"Number of Monthly Vehicle Collisions in New York City"}\hldef{,}
     \hlkwc{type}\hldef{=}\hlsng{"o"}\hldef{,}
     \hlkwc{pch}\hldef{=}\hlnum{20}\hldef{,}
     \hlkwc{xlab}\hldef{=}\hlsng{"Month"}\hldef{,}
     \hlkwc{ylab}\hldef{=}\hlsng{"Number of Collisions"}\hldef{,}
     \hlkwc{xaxt}\hldef{=}\hlsng{"n"}\hldef{,}
     \hlkwc{yaxt}\hldef{=}\hlsng{"n"}\hldef{,}
     \hlkwc{xaxs}\hldef{=}\hlsng{"i"}\hldef{,}
     \hlkwc{yaxs}\hldef{=}\hlsng{"i"}\hldef{,}
     \hlkwc{ylim}\hldef{=}\hlkwd{c}\hldef{(}\hlnum{0}\hldef{,}\hlkwd{max}\hldef{(colspermonth)}\hlopt{+}\hlnum{1000}\hldef{))}
\end{alltt}


{\ttfamily\noindent\bfseries\color{errorcolor}{\#\# Error: object 'plot\_months' not found}}\begin{alltt}
\hldef{xticks} \hlkwb{<-} \hlkwd{seq}\hldef{(}\hlkwd{min}\hldef{(plot_months),} \hlkwd{max}\hldef{(plot_months)}\hlopt{+}\hlnum{365}\hldef{,} \hlkwc{by} \hldef{=} \hlsng{"6 months"}\hldef{)}
\end{alltt}


{\ttfamily\noindent\bfseries\color{errorcolor}{\#\# Error: object 'plot\_months' not found}}\begin{alltt}
\hlkwd{axis.Date}\hldef{(}\hlnum{1}\hldef{,} \hlkwc{at} \hldef{= xticks,}
          \hlkwc{format} \hldef{=} \hlsng{"%b\textbackslash{}n%Y"}\hldef{,} \hlkwc{cex.axis} \hldef{=} \hlnum{0.6}\hldef{)}
\end{alltt}


{\ttfamily\noindent\bfseries\color{errorcolor}{\#\# Error: object 'xticks' not found}}\begin{alltt}
\hlkwd{axis}\hldef{(}\hlnum{2}\hldef{,} \hlkwc{at} \hldef{=} \hlkwd{seq}\hldef{(}\hlnum{0}\hldef{,} \hlkwd{max}\hldef{(colspermonth)}\hlopt{+}\hlnum{1000}\hldef{,} \hlkwc{by} \hldef{=} \hlnum{1000}\hldef{),} \hlkwc{cex.axis} \hldef{=} \hlnum{0.7}\hldef{,} \hlkwc{las}\hldef{=}\hlnum{1}\hldef{)}
\end{alltt}


{\ttfamily\noindent\bfseries\color{errorcolor}{\#\# Error: object 'colspermonth' not found}}\begin{alltt}
\hlkwd{abline}\hldef{(}\hlkwc{v} \hldef{= xticks,} \hlkwc{col} \hldef{=} \hlsng{"gray"}\hldef{,} \hlkwc{lty} \hldef{=} \hlsng{"dotted"}\hldef{)}
\end{alltt}


{\ttfamily\noindent\bfseries\color{errorcolor}{\#\# Error: object 'xticks' not found}}\end{kframe}
\end{knitrout}
A notable trend is the sudden drop in collisions at the beginning of 2020. This is clearly due to the Covid-19 pandemic, which limited the time people spent outside their homes. What is more notable is how the number of collisions seems to even out to a much lower level than pre-pandemic. More research is needed into why we see this pattern.
\\ \\
Next, we explore the number of collisions over time within each borough. Note that a large quantity of entries from the dataset do not have a location attached, so those results are not considered for this analysis.
\begin{knitrout}
\definecolor{shadecolor}{rgb}{0.969, 0.969, 0.969}\color{fgcolor}\begin{kframe}
\begin{alltt}
\hldef{borough} \hlkwb{<-} \hlkwd{levels}\hldef{(nyc}\hlopt{$}\hldef{BOROUGH)}
\end{alltt}


{\ttfamily\noindent\bfseries\color{errorcolor}{\#\# Error: object 'nyc' not found}}\begin{alltt}
\hldef{borough} \hlkwb{<-} \hldef{borough[borough} \hlopt{!=} \hlsng{""}\hldef{]}
\end{alltt}


{\ttfamily\noindent\bfseries\color{errorcolor}{\#\# Error: object 'borough' not found}}\begin{alltt}
\hldef{colors} \hlkwb{=} \hlkwd{c}\hldef{(}\hlsng{"red"}\hldef{,} \hlsng{"blue"}\hldef{,} \hlsng{"darkgreen"}\hldef{,} \hlsng{"purple"}\hldef{,} \hlsng{"orange"}\hldef{)}
\hldef{plot_months} \hlkwb{<-} \hlkwd{list}\hldef{()}
\hldef{colspermonth} \hlkwb{<-} \hlkwd{list}\hldef{()}
\hlkwa{for} \hldef{(b} \hlkwa{in} \hldef{borough)\{}
  \hldef{monthly_counts} \hlkwb{<-} \hlkwd{table}\hldef{(nyc}\hlopt{$}\hldef{YEAR_MONTH[nyc}\hlopt{$}\hldef{BOROUGH}\hlopt{==}\hldef{b])}
  \hldef{months} \hlkwb{<-} \hlkwd{names}\hldef{(monthly_counts)}
  \hldef{colspermonth[[b]]} \hlkwb{<-} \hlkwd{as.numeric}\hldef{(monthly_counts)}
  \hldef{plot_months[[b]]} \hlkwb{<-} \hlkwd{as.Date}\hldef{(}\hlkwd{paste0}\hldef{(months,} \hlsng{"-01"}\hldef{))}
\hldef{\}}
\end{alltt}


{\ttfamily\noindent\bfseries\color{errorcolor}{\#\# Error: object 'borough' not found}}\begin{alltt}
\hldef{max_col} \hlkwb{<-} \hlkwd{max}\hldef{(}\hlkwd{sapply}\hldef{(colspermonth, max))}
\end{alltt}


{\ttfamily\noindent\bfseries\color{errorcolor}{\#\# Error in max(sapply(colspermonth, max)): invalid 'type' (list) of argument}}\begin{alltt}
\hldef{min_dat} \hlkwb{<-} \hlkwd{as.Date}\hldef{(}\hlkwd{min}\hldef{(}\hlkwd{sapply}\hldef{(plot_months,min)))}
\end{alltt}


{\ttfamily\noindent\bfseries\color{errorcolor}{\#\# Error in min(sapply(plot\_months, min)): invalid 'type' (list) of argument}}\begin{alltt}
\hldef{max_dat} \hlkwb{<-} \hlkwd{as.Date}\hldef{(}\hlkwd{max}\hldef{(}\hlkwd{sapply}\hldef{(plot_months,max)))}
\end{alltt}


{\ttfamily\noindent\bfseries\color{errorcolor}{\#\# Error in max(sapply(plot\_months, max)): invalid 'type' (list) of argument}}\begin{alltt}
\hlkwd{plot}\hldef{(plot_months[[}\hlnum{1}\hldef{]], colspermonth[[}\hlnum{1}\hldef{]],}
     \hlkwc{main}\hldef{=}\hlsng{"Number of Monthly Vehicle Collisions in New York City"}\hldef{,}
     \hlkwc{type}\hldef{=}\hlsng{"l"}\hldef{,}
     \hlkwc{pch}\hldef{=}\hlnum{20}\hldef{,}
     \hlkwc{col}\hldef{= colors[}\hlnum{1}\hldef{],}
     \hlkwc{xlab}\hldef{=}\hlsng{"Month"}\hldef{,}
     \hlkwc{ylab}\hldef{=}\hlsng{"Number of Collisions"}\hldef{,}
     \hlkwc{xaxt}\hldef{=}\hlsng{"n"}\hldef{,}
     \hlkwc{yaxt}\hldef{=}\hlsng{"n"}\hldef{,}
     \hlkwc{xaxs}\hldef{=}\hlsng{"i"}\hldef{,}
     \hlkwc{yaxs}\hldef{=}\hlsng{"i"}\hldef{,}
     \hlkwc{ylim}\hldef{=}\hlkwd{c}\hldef{(}\hlnum{0}\hldef{,max_col}\hlopt{+}\hlnum{500}\hldef{))}
\end{alltt}


{\ttfamily\noindent\bfseries\color{errorcolor}{\#\# Error in plot\_months[[1]]: subscript out of bounds}}\begin{alltt}
\hlkwa{for} \hldef{(i} \hlkwa{in} \hlnum{2}\hlopt{:}\hldef{(}\hlkwd{length}\hldef{(borough)))\{}
  \hlkwd{lines}\hldef{(plot_months[[i]], colspermonth[[i]],} \hlkwc{type}\hldef{=}\hlsng{"l"}\hldef{,} \hlkwc{pch}\hldef{=}\hlnum{20}\hldef{,} \hlkwc{col}\hldef{=colors[i])}
\hldef{\}}
\end{alltt}


{\ttfamily\noindent\bfseries\color{errorcolor}{\#\# Error: object 'borough' not found}}\begin{alltt}
\hlkwd{legend}\hldef{(}\hlsng{"topright"}\hldef{,}
       \hlkwc{legend} \hldef{= borough,}
       \hlkwc{col} \hldef{= colors,}
       \hlkwc{lty} \hldef{=} \hlnum{1}\hldef{)}
\end{alltt}


{\ttfamily\noindent\bfseries\color{errorcolor}{\#\# Error: object 'borough' not found}}\begin{alltt}
\hldef{xticks} \hlkwb{<-} \hlkwd{seq}\hldef{(min_dat, max_dat}\hlopt{+}\hlnum{365}\hldef{,} \hlkwc{by} \hldef{=} \hlsng{"6 months"}\hldef{)}
\end{alltt}


{\ttfamily\noindent\bfseries\color{errorcolor}{\#\# Error: object 'min\_dat' not found}}\begin{alltt}
\hlkwd{axis.Date}\hldef{(}\hlnum{1}\hldef{,} \hlkwc{at} \hldef{= xticks,}
          \hlkwc{format} \hldef{=} \hlsng{"%b\textbackslash{}n%Y"}\hldef{,} \hlkwc{cex.axis} \hldef{=} \hlnum{0.6}\hldef{)}
\end{alltt}


{\ttfamily\noindent\bfseries\color{errorcolor}{\#\# Error: object 'xticks' not found}}\begin{alltt}
\hlkwd{axis}\hldef{(}\hlnum{2}\hldef{,} \hlkwc{at} \hldef{=} \hlkwd{seq}\hldef{(}\hlnum{0}\hldef{, max_col}\hlopt{+}\hlnum{1000}\hldef{,} \hlkwc{by} \hldef{=} \hlnum{1000}\hldef{),} \hlkwc{cex.axis} \hldef{=} \hlnum{0.7}\hldef{,} \hlkwc{las}\hldef{=}\hlnum{1}\hldef{)}
\end{alltt}


{\ttfamily\noindent\bfseries\color{errorcolor}{\#\# Error: object 'max\_col' not found}}\begin{alltt}
\hlkwd{abline}\hldef{(}\hlkwc{v} \hldef{= xticks,} \hlkwc{col} \hldef{=} \hlsng{"gray"}\hldef{,} \hlkwc{lty} \hldef{=} \hlsng{"dotted"}\hldef{)}
\end{alltt}


{\ttfamily\noindent\bfseries\color{errorcolor}{\#\# Error: object 'xticks' not found}}\end{kframe}
\end{knitrout}
Each borough follows very similar shapes in their respective line. Further analysis to normalize by some factor, e.g. by population or road network density may provide more insight.

\subsection{Causes of Collisions}
In the dataset every entry has up to 5 `contributing factors'. Below, We see these results aggregated and the top 5 most prevalent causes displayed in a bar chart.
\begin{knitrout}
\definecolor{shadecolor}{rgb}{0.969, 0.969, 0.969}\color{fgcolor}\begin{kframe}
\begin{alltt}
\hldef{cfs} \hlkwb{<-} \hlkwd{unique}\hldef{(nyc}\hlopt{$}\hldef{CONTRIBUTING.FACTOR.VEHICLE.1)}
\end{alltt}


{\ttfamily\noindent\bfseries\color{errorcolor}{\#\# Error: object 'nyc' not found}}\begin{alltt}
\hldef{cont_fact} \hlkwb{<-} \hlkwd{c}\hldef{()}
\hlkwa{for} \hldef{(f} \hlkwa{in} \hldef{cfs)\{}
  \hldef{s} \hlkwb{<-} \hlnum{0}
  \hlkwa{for} \hldef{(i} \hlkwa{in} \hlnum{1}\hlopt{:}\hlnum{5}\hldef{)\{}
    \hldef{colmn} \hlkwb{<-} \hlkwd{paste0}\hldef{(}\hlsng{"CONTRIBUTING.FACTOR.VEHICLE."}\hldef{,i)}
    \hldef{s} \hlkwb{<-} \hldef{s} \hlopt{+} \hlkwd{sum}\hldef{(nyc[[colmn]]} \hlopt{==} \hldef{f)}
  \hldef{\}}
  \hldef{cont_fact} \hlkwb{<-} \hlkwd{append}\hldef{(cont_fact,s)}
\hldef{\}}
\end{alltt}


{\ttfamily\noindent\bfseries\color{errorcolor}{\#\# Error: object 'cfs' not found}}\begin{alltt}
\hlkwd{names}\hldef{(cont_fact)} \hlkwb{<-} \hldef{cfs}
\end{alltt}


{\ttfamily\noindent\bfseries\color{errorcolor}{\#\# Error: object 'cfs' not found}}\begin{alltt}
\hldef{fivemost} \hlkwb{<-} \hlkwd{sort}\hldef{(cont_fact,} \hlkwc{decreasing}\hldef{=}\hlnum{TRUE}\hldef{)[}\hlnum{3}\hlopt{:}\hlnum{7}\hldef{]}
\hlcom{#1st and 2nd are unspecified or null}

\hlkwd{par}\hldef{(}\hlkwc{mar} \hldef{=} \hlkwd{c}\hldef{(}\hlnum{10}\hldef{,} \hlnum{5}\hldef{,} \hlnum{4}\hldef{,} \hlnum{2}\hldef{))}
\hlkwd{barplot}\hldef{(fivemost,}
        \hlkwc{main} \hldef{=} \hlsng{"Top Contributing Factors"}\hldef{,}
        \hlkwc{col} \hldef{=} \hlkwd{c}\hldef{(}\hlsng{"darkseagreen4"}\hldef{,} \hlsng{"darkseagreen"}\hldef{,} \hlsng{"darkseagreen3"}\hldef{,}
                \hlsng{"darkseagreen2"}\hldef{,} \hlsng{"darkseagreen1"}\hldef{),}
        \hlkwc{las} \hldef{=} \hlnum{2}\hldef{,}
        \hlkwc{cex.names} \hldef{=} \hlnum{0.8}\hldef{,}
        \hlkwc{yaxt} \hldef{=} \hlsng{"n"}\hldef{,}
        \hlkwc{ylim} \hldef{=} \hlkwd{c}\hldef{(}\hlnum{0}\hldef{,}\hlkwd{ceiling}\hldef{(}\hlkwd{max}\hldef{(fivemost)}\hlopt{/}\hlnum{100000}\hldef{)}\hlopt{*}\hlnum{100000}\hldef{)}
        \hldef{)}
\end{alltt}


{\ttfamily\noindent\bfseries\color{errorcolor}{\#\# Error in barplot.default(fivemost, main = "{}Top Contributing Factors"{}, : 'height' must be a vector or a matrix}}\begin{alltt}
\hlkwd{title}\hldef{(}\hlkwc{ylab} \hldef{=} \hlsng{"Number of Cases"}\hldef{,} \hlkwc{line}\hldef{=}\hlnum{4}\hldef{)}
\end{alltt}


{\ttfamily\noindent\bfseries\color{errorcolor}{\#\# Error in title(ylab = "{}Number of Cases"{}, line = 4): plot.new has not been called yet}}\begin{alltt}
\hlkwd{axis}\hldef{(}\hlnum{2}\hldef{,} \hlkwc{at} \hldef{=} \hlkwd{seq}\hldef{(}\hlnum{0}\hldef{,} \hlkwd{ceiling}\hldef{(}\hlkwd{max}\hldef{(fivemost)}\hlopt{/}\hlnum{100000}\hldef{)}\hlopt{*}\hlnum{100000}\hldef{,} \hlkwc{by} \hldef{=} \hlnum{100000}\hldef{),}
     \hlkwc{labels} \hldef{=} \hlkwd{format}\hldef{(}\hlkwd{seq}\hldef{(}\hlnum{0}\hldef{,} \hlkwd{ceiling}\hldef{(}\hlkwd{max}\hldef{(fivemost)}\hlopt{/}\hlnum{100000}\hldef{)}\hlopt{*}\hlnum{100000}\hldef{,} \hlkwc{by} \hldef{=} \hlnum{100000}\hldef{),}
                     \hlkwc{big.mark} \hldef{=} \hlsng{","}\hldef{,}
                     \hlkwc{scientific} \hldef{=} \hlnum{FALSE}\hldef{),}
     \hlkwc{cex.axis} \hldef{=} \hlnum{0.8}\hldef{,}
     \hlkwc{las} \hldef{=} \hlnum{1}\hldef{,}
     \hldef{)}
\end{alltt}


{\ttfamily\noindent\color{warningcolor}{\#\# Warning in max(fivemost): no non-missing arguments to max; returning -Inf}}

{\ttfamily\noindent\bfseries\color{errorcolor}{\#\# Error in seq.default(0, ceiling(max(fivemost)/1e+05) * 1e+05, by = 1e+05): 'to' must be a finite number}}\end{kframe}
\end{knitrout}
An overwhelming cause of accidents are due to driver distraction, more than 3 times the next most common.


\subsection{Crashes by Time of Day}

The histogram below shows the total number of accidents in the dataset for each time of day. 

\begin{knitrout}
\definecolor{shadecolor}{rgb}{0.969, 0.969, 0.969}\color{fgcolor}\begin{kframe}
\begin{alltt}
\hlkwd{layout}\hldef{(}\hlkwc{mat} \hldef{=} \hlnum{1}\hldef{)}
\hldef{times} \hlkwb{<-} \hldef{nyc}\hlopt{$}\hldef{CRASH.TIME}
\end{alltt}


{\ttfamily\noindent\bfseries\color{errorcolor}{\#\# Error: object 'nyc' not found}}\begin{alltt}
\hldef{hours} \hlkwb{<-} \hlkwd{as.numeric}\hldef{(}\hlkwd{sub}\hldef{(}\hlsng{"\textbackslash{}\textbackslash{}:.*"}\hldef{,} \hlsng{""}\hldef{, times))}
\end{alltt}


{\ttfamily\noindent\bfseries\color{errorcolor}{\#\# Error: object 'times' not found}}\begin{alltt}
\hlkwd{par}\hldef{(}\hlkwc{tck} \hldef{=} \hlopt{-}\hlnum{0.015}\hldef{,} \hlkwc{mgp} \hldef{=} \hlkwd{c}\hldef{(}\hlnum{1.5}\hldef{,} \hlnum{0.4}\hldef{,} \hlnum{0}\hldef{),} \hlkwc{mar} \hldef{=} \hlkwd{c}\hldef{(}\hlnum{3}\hldef{,} \hlnum{3}\hldef{,} \hlnum{3}\hldef{,} \hlnum{1.5}\hldef{))}
\hlkwd{hist}\hldef{(hours,} \hlkwc{xlab} \hldef{=} \hlsng{"Time of day"}\hldef{,} \hlkwc{ylab} \hldef{=} \hlsng{"Number of accidents"}\hldef{,} \hlkwc{axes} \hldef{=} \hlnum{FALSE}\hldef{,}
     \hlkwc{main}\hldef{=}\hlsng{"Total Number of Accidents per Time of Day"}\hldef{,}
     \hlkwc{breaks}\hldef{=}\hlkwd{seq}\hldef{(}\hlopt{-}\hlnum{0.5}\hldef{,} \hlnum{23.5}\hldef{,} \hlnum{1}\hldef{))}
\end{alltt}


{\ttfamily\noindent\bfseries\color{errorcolor}{\#\# Error: object 'hours' not found}}\begin{alltt}
\hldef{labels} \hlkwb{<-} \hlkwd{c}\hldef{(}\hlsng{"12:00AM"}\hldef{,} \hlsng{"3:00AM"}\hldef{,} \hlsng{"6:00AM"}\hldef{,} \hlsng{"9:00AM"}\hldef{,}
            \hlsng{"12:00PM"}\hldef{,} \hlsng{"3:00PM"}\hldef{,} \hlsng{"6:00PM"}\hldef{,} \hlsng{"9:00PM"}\hldef{)}
\hldef{axp} \hlkwb{<-} \hlkwd{seq}\hldef{(}\hlnum{0}\hldef{,} \hlnum{23}\hldef{,} \hlkwc{by}\hldef{=}\hlnum{3}\hldef{)}
\hlkwd{axis}\hldef{(}\hlnum{1}\hldef{,} \hlkwc{at}\hldef{=axp,} \hlkwc{labels} \hldef{= labels)}
\end{alltt}


{\ttfamily\noindent\bfseries\color{errorcolor}{\#\# Error in axis(1, at = axp, labels = labels): plot.new has not been called yet}}\begin{alltt}
\hldef{axp2} \hlkwb{<-} \hlkwd{seq}\hldef{(}\hlnum{0}\hldef{,} \hlnum{150000}\hldef{,} \hlkwc{by}\hldef{=}\hlnum{50000}\hldef{)}
\hlkwd{axis}\hldef{(}\hlnum{2}\hldef{,} \hlkwc{at} \hldef{= axp2,} \hlkwc{labels} \hldef{= axp2)}
\end{alltt}


{\ttfamily\noindent\bfseries\color{errorcolor}{\#\# Error in axis(2, at = axp2, labels = axp2): plot.new has not been called yet}}\end{kframe}
\end{knitrout}

We can see that most of the accidents occur during the daytime, with a semi-continuous increasing and decreasing structure. The distribution reaches its lowest value at 3:00AM, and peaks around 4:00PM. There are slight spikes around 8-9AM, when people would be driving to work, and also around 4-5:00PM, when people would be leaving work. There is also a small spike around midnight.

\subsection{Injuries by Person Type}

This series of box plots focuses on collisions that resulted in at least 10 total injuries. We compare the total number of people injured with the amounts of people injured of different types: pedestrians, motorists, and cyclists.

\begin{knitrout}
\definecolor{shadecolor}{rgb}{0.969, 0.969, 0.969}\color{fgcolor}\begin{kframe}
\begin{alltt}
\hldef{people_injured} \hlkwb{<-} \hldef{nyc}\hlopt{$}\hldef{NUMBER.OF.PERSONS.INJURED}
\end{alltt}


{\ttfamily\noindent\bfseries\color{errorcolor}{\#\# Error: object 'nyc' not found}}\begin{alltt}
\hldef{pedestrians_injured} \hlkwb{<-} \hldef{nyc}\hlopt{$}\hldef{NUMBER.OF.PEDESTRIANS.INJURED}
\end{alltt}


{\ttfamily\noindent\bfseries\color{errorcolor}{\#\# Error: object 'nyc' not found}}\begin{alltt}
\hldef{motorists_injured} \hlkwb{<-} \hldef{nyc}\hlopt{$}\hldef{NUMBER.OF.MOTORIST.INJURED}
\end{alltt}


{\ttfamily\noindent\bfseries\color{errorcolor}{\#\# Error: object 'nyc' not found}}\begin{alltt}
\hldef{cyclists_injured} \hlkwb{<-} \hldef{nyc}\hlopt{$}\hldef{NUMBER.OF.CYCLIST.INJURED}
\end{alltt}


{\ttfamily\noindent\bfseries\color{errorcolor}{\#\# Error: object 'nyc' not found}}\begin{alltt}
\hldef{mass_factor} \hlkwb{<-} \hldef{people_injured} \hlopt{>=} \hlnum{10}
\end{alltt}


{\ttfamily\noindent\bfseries\color{errorcolor}{\#\# Error: object 'people\_injured' not found}}\begin{alltt}
\hldef{people_injured_mass} \hlkwb{<-} \hldef{people_injured[mass_factor]}
\end{alltt}


{\ttfamily\noindent\bfseries\color{errorcolor}{\#\# Error: object 'people\_injured' not found}}\begin{alltt}
\hldef{motorists_injured_mass} \hlkwb{<-} \hldef{motorists_injured[mass_factor]}
\end{alltt}


{\ttfamily\noindent\bfseries\color{errorcolor}{\#\# Error: object 'motorists\_injured' not found}}\begin{alltt}
\hldef{pedestrians_injured_mass} \hlkwb{<-} \hldef{pedestrians_injured[mass_factor]}
\end{alltt}


{\ttfamily\noindent\bfseries\color{errorcolor}{\#\# Error: object 'pedestrians\_injured' not found}}\begin{alltt}
\hldef{cyclists_injured_mass} \hlkwb{<-} \hldef{cyclists_injured[mass_factor]}
\end{alltt}


{\ttfamily\noindent\bfseries\color{errorcolor}{\#\# Error: object 'cyclists\_injured' not found}}\begin{alltt}
\hlkwd{layout}\hldef{(}\hlkwc{mat} \hldef{=} \hlkwd{matrix}\hldef{(}\hlkwd{c}\hldef{(}\hlnum{1}\hldef{,} \hlnum{2}\hldef{,} \hlnum{3}\hldef{,} \hlnum{4}\hldef{),} \hlnum{4}\hldef{,} \hlkwc{byrow} \hldef{=} \hlnum{TRUE}\hldef{))}
\hlkwd{boxplot}\hldef{(people_injured_mass,}
        \hlkwc{main}\hldef{=}\hlsng{"total people injured for mass casualty events 
        (over 10 people injured)"}\hldef{,}
        \hlkwc{horizontal}\hldef{=}\hlnum{TRUE}\hldef{,} \hlkwc{ylim} \hldef{=} \hlkwd{c}\hldef{(}\hlnum{0}\hldef{,} \hlnum{45}\hldef{))}
\end{alltt}


{\ttfamily\noindent\bfseries\color{errorcolor}{\#\# Error: object 'people\_injured\_mass' not found}}\begin{alltt}
\hlkwd{boxplot}\hldef{(motorists_injured_mass,} \hlkwc{main}\hldef{=}\hlsng{"motorists injured"}\hldef{,}
        \hlkwc{horizontal}\hldef{=}\hlnum{TRUE}\hldef{,} \hlkwc{ylim} \hldef{=} \hlkwd{c}\hldef{(}\hlnum{0}\hldef{,} \hlnum{45}\hldef{))}
\end{alltt}


{\ttfamily\noindent\bfseries\color{errorcolor}{\#\# Error: object 'motorists\_injured\_mass' not found}}\begin{alltt}
\hlkwd{boxplot}\hldef{(pedestrians_injured_mass,} \hlkwc{main}\hldef{=}\hlsng{"pedestrians injured"}\hldef{,}
        \hlkwc{horizontal}\hldef{=}\hlnum{TRUE}\hldef{,} \hlkwc{ylim} \hldef{=} \hlkwd{c}\hldef{(}\hlnum{0}\hldef{,} \hlnum{45}\hldef{))}
\end{alltt}


{\ttfamily\noindent\bfseries\color{errorcolor}{\#\# Error: object 'pedestrians\_injured\_mass' not found}}\begin{alltt}
\hlkwd{boxplot}\hldef{(cyclists_injured_mass,} \hlkwc{main} \hldef{=} \hlsng{"cyclists injured"}\hldef{,}
        \hlkwc{horizontal}\hldef{=}\hlnum{TRUE}\hldef{,} \hlkwc{ylim} \hldef{=} \hlkwd{c}\hldef{(}\hlnum{0}\hldef{,} \hlnum{45}\hldef{))}
\end{alltt}


{\ttfamily\noindent\bfseries\color{errorcolor}{\#\# Error: object 'cyclists\_injured\_mass' not found}}\end{kframe}
\end{knitrout}

The vast majority of these collisions resulted in motorists being the main affected group. This is unsurprising, as every collision must involve motorists, but the other possible parties may not even be present. Pedestrians were mostly uninjured by these collisions, with some exceptions that result in the amount of pedestrian injuries to go into the 20s. Cyclists were the most unaffected by these mass injury collisions, with a couple of collisions injuring one cyclist, but otherwise 0 cyclists were injured in any of the other collisions. 


\subsection{Borough VS Injury Severity}

\subsubsection{Contingency Table}

This contingency table examines whether crash severity (fatal, injury, or no injury) varies across NYC boroughs.

\begin{knitrout}
\definecolor{shadecolor}{rgb}{0.969, 0.969, 0.969}\color{fgcolor}\begin{kframe}
\begin{alltt}
\hldef{nyc}\hlopt{$}\hldef{injury_severity} \hlkwb{<-} \hlkwd{ifelse}\hldef{(nyc}\hlopt{$}\hldef{NUMBER.OF.CYCLIST.KILLED} \hlopt{>} \hlnum{0}\hldef{,}
                              \hlsng{"FATAL/DIED"}\hldef{,}
                       \hlkwd{ifelse}\hldef{(nyc}\hlopt{$}\hldef{NUMBER.OF.PERSONS.INJURED} \hlopt{>} \hlnum{0}\hldef{,}
                              \hlsng{"INJURY"}\hldef{,} \hlsng{"NO INJURY"}\hldef{))}
\end{alltt}


{\ttfamily\noindent\bfseries\color{errorcolor}{\#\# Error: object 'nyc' not found}}\begin{alltt}
\hlcom{# removing NA's}
\hldef{borough_injury_data} \hlkwb{<-} \hldef{nyc[}\hlopt{!}\hlkwd{is.na}\hldef{(nyc}\hlopt{$}\hldef{BOROUGH)} \hlopt{& !}\hlkwd{is.na}\hldef{(nyc}\hlopt{$}\hldef{injury_severity), ]}
\end{alltt}


{\ttfamily\noindent\bfseries\color{errorcolor}{\#\# Error: object 'nyc' not found}}\begin{alltt}
\hlcom{# Contingency Table}
\hldef{(borough_injury_table} \hlkwb{<-} \hlkwd{table}\hldef{(borough_injury_data}\hlopt{$}\hldef{BOROUGH,}
                               \hldef{borough_injury_data}\hlopt{$}\hldef{injury_severity))}
\end{alltt}


{\ttfamily\noindent\bfseries\color{errorcolor}{\#\# Error: object 'borough\_injury\_data' not found}}\end{kframe}
\end{knitrout}

\begin{knitrout}
\definecolor{shadecolor}{rgb}{0.969, 0.969, 0.969}\color{fgcolor}\begin{kframe}
\begin{alltt}
\hlcom{# Chi-square test}
\hldef{(chi_test1} \hlkwb{<-} \hlkwd{chisq.test}\hldef{(borough_injury_table))}
\end{alltt}


{\ttfamily\noindent\bfseries\color{errorcolor}{\#\# Error: object 'borough\_injury\_table' not found}}\end{kframe}
\end{knitrout}

Since p-value < alpha = 0.05, we reject the null hypothesis and conclude that we have sufficient evidence to prove the true population of boroughs are not evenly distributed or in other words, that the distribution of fatal, injury, and property-damage-only crashes varies significantly across NYC boroughs (Chi-square, df=8, p-value < 0.0001). Furthermore, there is a massive test statistic of $X^2$ = 6377.3 which shows the huge differences between what we'd expect if boroughs were all the same vs. what we actually observe.


\subsubsection{Barplot of Row Proportions (by Borough)}
This barplot shows crash severity proportions within each borough.

\begin{knitrout}
\definecolor{shadecolor}{rgb}{0.969, 0.969, 0.969}\color{fgcolor}\begin{kframe}
\begin{alltt}
\hldef{(prop_table1} \hlkwb{<-} \hlkwd{prop.table}\hldef{(borough_injury_table,} \hlkwc{margin} \hldef{=} \hlnum{1}\hldef{))}
\end{alltt}


{\ttfamily\noindent\bfseries\color{errorcolor}{\#\# Error: object 'borough\_injury\_table' not found}}\begin{alltt}
\hlkwd{barplot}\hldef{(prop_table1)}
\end{alltt}


{\ttfamily\noindent\bfseries\color{errorcolor}{\#\# Error: object 'prop\_table1' not found}}\end{kframe}
\end{knitrout}

While all boroughs have low fatal crash rates (around 0.1-0.16\%), Staten Island has the highest fatality rate at 0.159\%, followed by Brooklyn at 0.142\%. Manhattan appears to be the safest borough with the lowest rates of both fatal (0.107\%) and injury crashes (18.9\%). Brooklyn and the Bronx have the highest injury rates at 26.1\% and 25.3\% respectively, while Manhattan has significantly more no injury crashes (80.1\%) compared to other boroughs.

\subsubsection{Cramer’s V}



\begin{knitrout}
\definecolor{shadecolor}{rgb}{0.969, 0.969, 0.969}\color{fgcolor}\begin{kframe}
\begin{alltt}
\hlkwd{cramerV}\hldef{(borough_injury_table)}
\end{alltt}


{\ttfamily\noindent\bfseries\color{errorcolor}{\#\# Error: object 'borough\_injury\_table' not found}}\end{kframe}
\end{knitrout}

The chi-square test concluded that there is a statistically significant association between borough and crash severity. However, the effect size was very small (Cramer's V = 0.046), indicating that while borough differences are statistically detectable, they explain less than 1\% of the variation in crash severity. This suggests that factors other than borough location are much more important in determining crash outcomes.

\subsection{Total Persons Injured vs Killed}
This scatterplot examines the relationship between injuries and fatalities in NYC vehicle crashes.
\begin{knitrout}
\definecolor{shadecolor}{rgb}{0.969, 0.969, 0.969}\color{fgcolor}\begin{kframe}
\begin{alltt}
\hldef{injury_data} \hlkwb{<-} \hldef{nyc[nyc}\hlopt{$}\hldef{NUMBER.OF.PERSONS.INJURED} \hlopt{>} \hlnum{0} \hlopt{|}
                     \hldef{nyc}\hlopt{$}\hldef{NUMBER.OF.PERSONS.KILLED} \hlopt{>} \hlnum{0}\hldef{, ]}
\end{alltt}


{\ttfamily\noindent\bfseries\color{errorcolor}{\#\# Error: object 'nyc' not found}}\begin{alltt}
\hlkwd{plot}\hldef{(injury_data}\hlopt{$}\hldef{NUMBER.OF.PERSONS.INJURED,}
     \hldef{injury_data}\hlopt{$}\hldef{NUMBER.OF.PERSONS.KILLED,}
     \hlkwc{main} \hldef{=} \hlsng{"Relationship Between Injuries and Fatalities in NYC Crashes"}\hldef{,}
     \hlkwc{xlab} \hldef{=} \hlsng{"Number of Persons Injured"}\hldef{,}
     \hlkwc{ylab} \hldef{=} \hlsng{"Number of Persons Killed"}\hldef{,}
     \hlkwc{pch} \hldef{=} \hlnum{16}\hldef{,}
     \hlkwc{col} \hldef{=} \hlsng{"darkblue"}\hldef{,}
     \hlkwc{cex} \hldef{=} \hlnum{0.6}\hldef{)}
\end{alltt}


{\ttfamily\noindent\bfseries\color{errorcolor}{\#\# Error: object 'injury\_data' not found}}\begin{alltt}
\hlkwd{abline}\hldef{(}\hlkwd{lm}\hldef{(injury_data}\hlopt{$}\hldef{NUMBER.OF.PERSONS.KILLED} \hlopt{~}
            \hldef{injury_data}\hlopt{$}\hldef{NUMBER.OF.PERSONS.INJURED),}
       \hlkwc{col} \hldef{=} \hlsng{"red"}\hldef{,} \hlkwc{lwd} \hldef{=} \hlnum{2}\hldef{)}
\end{alltt}


{\ttfamily\noindent\bfseries\color{errorcolor}{\#\# Error in eval(predvars, data, env): object 'injury\_data' not found}}\end{kframe}
\end{knitrout}

The plot shows that most crashes result in injuries but no deaths which can be seen by the concentration of points along the bottom. Fatal crashes are rare and scattered across injury levels, with no clear correlation. The flat regression line shows that there is a minimal relationship between injuries and deaths, suggesting fatalities depend more on crash circumstances than the number of people involved.



\section{Preliminary Statistical Modeling \& Interpretation}

To better understand the factors that influence the severity of motor vehicle collisions in New York City, we developed a preliminary statistical model using logistic regression. The outcome of interest is whether a crash was fatal (at least one person killed) or non-fatal. Logistic regression is appropriate in this context because the outcome is binary (fatal vs. non-fatal), and it allows us to estimate how different factors change the odds of a fatal crash while holding other conditions constant.

The predictors included in the model were the borough of occurrence, time of day (day vs. night), vehicle type, and the primary contributing factor reported by police. By examining the estimated odds ratios and their statistical significance, we can identify which circumstances or behaviors are most strongly associated with fatal outcomes.

\subsection{Code and Results}



Data cleansing and transformations:
\begin{knitrout}
\definecolor{shadecolor}{rgb}{0.969, 0.969, 0.969}\color{fgcolor}\begin{kframe}
\begin{alltt}
\hldef{nyc_small} \hlkwb{<-} \hldef{nyc} \hlopt
  \hlkwd{filter}\hldef{(BOROUGH} \hlopt{!=} \hlsng{""} \hlopt{&}
           \hldef{CONTRIBUTING.FACTOR.VEHICLE.1} \hlopt{!=} \hlsng{""} \hlopt{&}
           \hldef{CONTRIBUTING.FACTOR.VEHICLE.1} \hlopt{!=} \hlsng{"1"} \hlopt{&}
           \hldef{CONTRIBUTING.FACTOR.VEHICLE.1} \hlopt{!=} \hlsng{"80"}\hldef{)} \hlopt
  \hlkwd{transmute}\hldef{(}
    \hlkwc{is_fatal} \hldef{=} \hlkwd{ifelse}\hldef{(NUMBER.OF.PERSONS.KILLED} \hlopt{>} \hlnum{0}\hldef{,} \hlnum{1}\hldef{,} \hlnum{0}\hldef{),}  \hlcom{# outcome}
    \hlkwc{borough} \hldef{= BOROUGH,}
    \hlkwc{crash_hour} \hldef{=} \hlkwd{hour}\hldef{(}\hlkwd{hm}\hldef{(CRASH.TIME)),}
    \hlkwc{contributing_factor} \hldef{= CONTRIBUTING.FACTOR.VEHICLE.1,}
    \hlkwc{vehicle_cat} \hldef{=} \hlkwd{factor}\hldef{(}\hlkwd{case_when}\hldef{(}
      \hlkwd{grepl}\hldef{(}\hlsng{"MOTORCYCLE|MOTORBIKE"}\hldef{, VEHICLE.TYPE.CODE.1,} \hlkwc{ignore.case} \hldef{=} \hlnum{TRUE}\hldef{)} \hlopt{~}
        \hlsng{"motorcycle"}\hldef{,}
      \hlkwd{grepl}\hldef{(}\hlsng{"TRUCK|BUS"}\hldef{, VEHICLE.TYPE.CODE.1,} \hlkwc{ignore.case} \hldef{=} \hlnum{TRUE}\hldef{)} \hlopt{~}
        \hlsng{"truck_bus"}\hldef{,}
      \hlkwd{grepl}\hldef{(}\hlsng{"BICYCLE|BIKE|SCOOTER"}\hldef{, VEHICLE.TYPE.CODE.1,} \hlkwc{ignore.case} \hldef{=} \hlnum{TRUE}\hldef{)} \hlopt{~}
        \hlsng{"bike_scooter"}\hldef{,}
      \hlnum{TRUE} \hlopt{~} \hlsng{"car_suv"}
    \hldef{))}
  \hldef{)}
\end{alltt}


{\ttfamily\noindent\bfseries\color{errorcolor}{\#\# Error: object 'nyc' not found}}\begin{alltt}
\hldef{nyc_small} \hlkwb{<-} \hldef{nyc_small} \hlopt
  \hlkwd{mutate}\hldef{(}\hlkwc{hour} \hldef{=} \hlkwd{as.numeric}\hldef{(}\hlkwd{substr}\hldef{(crash_hour,} \hlnum{1}\hldef{,} \hlnum{2}\hldef{)),}
         \hlkwc{tod} \hldef{=} \hlkwd{ifelse}\hldef{(hour} \hlopt{>=} \hlnum{6} \hlopt{&} \hldef{hour} \hlopt{<} \hlnum{18}\hldef{,} \hlsng{"day"}\hldef{,} \hlsng{"night"}\hldef{))}
\end{alltt}


{\ttfamily\noindent\bfseries\color{errorcolor}{\#\# Error: object 'nyc\_small' not found}}\begin{alltt}
\hldef{nyc_small}\hlopt{$}\hldef{BOROUGH} \hlkwb{<-} \hlkwd{droplevels}\hldef{(}\hlkwd{as.factor}\hldef{(nyc_small}\hlopt{$}\hldef{borough))}
\end{alltt}


{\ttfamily\noindent\bfseries\color{errorcolor}{\#\# Error: object 'nyc\_small' not found}}\begin{alltt}
\hlcom{# set baselines for model}
\hldef{nyc_small}\hlopt{$}\hldef{borough} \hlkwb{<-} \hlkwd{relevel}\hldef{(}\hlkwd{as.factor}\hldef{(nyc_small}\hlopt{$}\hldef{BOROUGH),}
                             \hlkwc{ref} \hldef{=} \hlsng{"STATEN ISLAND"}\hldef{)}
\end{alltt}


{\ttfamily\noindent\bfseries\color{errorcolor}{\#\# Error: object 'nyc\_small' not found}}\begin{alltt}
\hldef{nyc_small}\hlopt{$}\hldef{tod} \hlkwb{<-} \hlkwd{relevel}\hldef{(}\hlkwd{as.factor}\hldef{(nyc_small}\hlopt{$}\hldef{tod),}
                         \hlkwc{ref} \hldef{=} \hlsng{"day"}\hldef{)}
\end{alltt}


{\ttfamily\noindent\bfseries\color{errorcolor}{\#\# Error: object 'nyc\_small' not found}}\begin{alltt}
\hldef{nyc_small}\hlopt{$}\hldef{vehicle_cat} \hlkwb{<-} \hlkwd{relevel}\hldef{(}\hlkwd{as.factor}\hldef{(nyc_small}\hlopt{$}\hldef{vehicle_cat),}
                                 \hlkwc{ref} \hldef{=} \hlsng{"car_suv"}\hldef{)}
\end{alltt}


{\ttfamily\noindent\bfseries\color{errorcolor}{\#\# Error: object 'nyc\_small' not found}}\begin{alltt}
\hldef{nyc_small}\hlopt{$}\hldef{contributing_factor} \hlkwb{<-} \hlkwd{relevel}\hldef{(}
  \hlkwd{as.factor}\hldef{(nyc_small}\hlopt{$}\hldef{contributing_factor),}\hlkwc{ref} \hldef{=} \hlsng{"Unspecified"}\hldef{)}
\end{alltt}


{\ttfamily\noindent\bfseries\color{errorcolor}{\#\# Error: object 'nyc\_small' not found}}\end{kframe}
\end{knitrout}

Creating the logistic regression model:
\begin{knitrout}
\definecolor{shadecolor}{rgb}{0.969, 0.969, 0.969}\color{fgcolor}\begin{kframe}
\begin{alltt}
\hldef{model} \hlkwb{<-} \hlkwd{glm}\hldef{(is_fatal} \hlopt{~} \hldef{borough} \hlopt{+} \hldef{tod} \hlopt{+} \hldef{vehicle_cat} \hlopt{+} \hldef{contributing_factor,}
             \hlkwc{data} \hldef{= nyc_small,}
             \hlkwc{family} \hldef{= binomial)}
\end{alltt}


{\ttfamily\noindent\bfseries\color{errorcolor}{\#\# Error in eval(mf, parent.frame()): object 'nyc\_small' not found}}\begin{alltt}
\hldef{summ} \hlkwb{<-} \hlkwd{summary}\hldef{(model)}
\end{alltt}


{\ttfamily\noindent\bfseries\color{errorcolor}{\#\# Error: object 'model' not found}}\begin{alltt}
\hldef{coefs} \hlkwb{<-} \hldef{summ}\hlopt{$}\hldef{coefficients}
\end{alltt}


{\ttfamily\noindent\bfseries\color{errorcolor}{\#\# Error: object 'summ' not found}}\begin{alltt}
\hldef{est} \hlkwb{<-} \hldef{coefs[,} \hlsng{"Estimate"}\hldef{]}         \hlcom{# log-odds}
\end{alltt}


{\ttfamily\noindent\bfseries\color{errorcolor}{\#\# Error: object 'coefs' not found}}\begin{alltt}
\hldef{se}  \hlkwb{<-} \hldef{coefs[,} \hlsng{"Std. Error"}\hldef{]}       \hlcom{# standard errors}
\end{alltt}


{\ttfamily\noindent\bfseries\color{errorcolor}{\#\# Error: object 'coefs' not found}}\begin{alltt}
\hlcom{# compute odds ratios + 95% confidence intervals}
\hldef{ci_low}  \hlkwb{<-} \hlkwd{exp}\hldef{(est} \hlopt{-} \hlnum{1.96} \hlopt{*} \hldef{se)}
\end{alltt}


{\ttfamily\noindent\bfseries\color{errorcolor}{\#\# Error: object 'est' not found}}\begin{alltt}
\hldef{ci_high} \hlkwb{<-} \hlkwd{exp}\hldef{(est} \hlopt{+} \hlnum{1.96} \hlopt{*} \hldef{se)}
\end{alltt}


{\ttfamily\noindent\bfseries\color{errorcolor}{\#\# Error: object 'est' not found}}\begin{alltt}
\hldef{results_table} \hlkwb{<-} \hlkwd{data.frame}\hldef{(}
  \hlkwc{term}        \hldef{=} \hlkwd{rownames}\hldef{(coefs),}
  \hlkwc{odds_ratio}  \hldef{=} \hlkwd{exp}\hldef{(est),}
  \hlkwc{ci_low}      \hldef{= ci_low,}
  \hlkwc{ci_high}     \hldef{= ci_high,}
  \hlkwc{significant} \hldef{=} \hlkwd{ifelse}\hldef{(coefs[,} \hlsng{"Pr(>|z|)"}\hldef{]} \hlopt{<} \hlnum{0.05}\hldef{,} \hlsng{"yes"}\hldef{,} \hlsng{"no"}\hldef{),}
  \hlkwc{row.names}   \hldef{=} \hlkwa{NULL}
\hldef{)}
\end{alltt}


{\ttfamily\noindent\bfseries\color{errorcolor}{\#\# Error: object 'coefs' not found}}\begin{alltt}
\hlcom{# filter out the intercept/baseline to keep table cleaner}
\hldef{results_table} \hlkwb{<-} \hldef{results_table[results_table}\hlopt{$}\hldef{term} \hlopt{!=} \hlsng{"(Intercept)"}\hldef{, ]}
\end{alltt}


{\ttfamily\noindent\bfseries\color{errorcolor}{\#\# Error: object 'results\_table' not found}}\begin{alltt}
\hldef{results_table}\hlopt{$}\hldef{term} \hlkwb{<-} \hlkwd{gsub}\hldef{(}\hlsng{"vehicle_cat"}\hldef{,} \hlsng{"Vehicle: "}\hldef{,}
                           \hldef{results_table}\hlopt{$}\hldef{term)}
\end{alltt}


{\ttfamily\noindent\bfseries\color{errorcolor}{\#\# Error: object 'results\_table' not found}}\begin{alltt}
\hldef{results_table}\hlopt{$}\hldef{term} \hlkwb{<-} \hlkwd{gsub}\hldef{(}\hlsng{"borough"}\hldef{,} \hlsng{"Borough: "}\hldef{,}
                           \hldef{results_table}\hlopt{$}\hldef{term)}
\end{alltt}


{\ttfamily\noindent\bfseries\color{errorcolor}{\#\# Error: object 'results\_table' not found}}\begin{alltt}
\hldef{results_table}\hlopt{$}\hldef{term} \hlkwb{<-} \hlkwd{gsub}\hldef{(}\hlsng{"tod"}\hldef{,} \hlsng{"Time of Day: "}\hldef{,}
                           \hldef{results_table}\hlopt{$}\hldef{term)}
\end{alltt}


{\ttfamily\noindent\bfseries\color{errorcolor}{\#\# Error: object 'results\_table' not found}}\begin{alltt}
\hldef{results_table}\hlopt{$}\hldef{term} \hlkwb{<-} \hlkwd{gsub}\hldef{(}\hlsng{"contributing_factor"}\hldef{,} \hlsng{"Contributing Factor: "}\hldef{,}
                           \hldef{results_table}\hlopt{$}\hldef{term)}
\end{alltt}


{\ttfamily\noindent\bfseries\color{errorcolor}{\#\# Error: object 'results\_table' not found}}\begin{alltt}
\hldef{results_table}
\end{alltt}


{\ttfamily\noindent\bfseries\color{errorcolor}{\#\# Error: object 'results\_table' not found}}\end{kframe}
\end{knitrout}

\subsection{Interpretation}
The revised model confirms several strong predictors of fatal crashes. Compared to Staten Island, crashes in the Bronx, Queens, and Manhattan show significantly lower odds of being fatal, while Brooklyn is not statistically different. Nighttime crashes remain much riskier, with nearly 1.8 times higher odds of fatality compared to daytime. Vehicle type shows especially pronounced differences: motorcycle crashes are about 14 times more likely to be fatal than car/SUV crashes, while trucks and buses carry over 3 times the risk and bicycles/scooters nearly 3 times the risk.

Reported contributing factors highlight behaviors that dramatically elevate risk. Crashes linked to alcohol are almost 2 times more likely to result in death, while those involving drug use show odds 8–10 times higher than the baseline. Dangerous driving behaviors such as speeding, traffic signal violations, unsafe lane changes, and pedestrian/bicyclist error all significantly increase fatal crash likelihood, often by a factor of 4–5 or more. By contrast, mechanical defect categories (e.g., “brakes defective”) produce unstable estimates due to rare occurrences and should be interpreted cautiously.

Overall, these results suggest that time of day, vehicle type, and risky human behaviors are the dominant predictors of fatal outcomes in NYC crashes, while borough differences are smaller in magnitude.

\end{document}
